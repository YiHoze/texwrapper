% C:\>ltx.py -L -w -P hanroman.tex

\documentclass{article}
\usepackage{kotex}
\usepackage{pythontex}
\setmainhangulfont{Noto Serif KR}

\ExplSyntaxOn
\sys_if_engine_xetex:T
{
  \usepackage{ksruby}
  \cs_set_eq:NN \ruby \ksruby
}
\ExplSyntaxOff

\begin{pycode}
import sys
sys.path.append('c:/home/bin')
from hanroman import *
K2LVT=KoreanToRoman(unravel=True)
K2R=KoreanToRoman()
E2H=EnglishToHangul()
\end{pycode}

\NewDocumentCommand\kortoLVT{m}{\py{K2LVT.transcribe("#1", special=True, separator=False, capital=False, transliterate=False)}}
\NewDocumentCommand\kortorom{m}{\py{K2R.transcribe("#1", special=False, separator=False, capital=False, transliterate=False)}}
\NewDocumentCommand\engtohan{m}{\py{E2H.transcribe("#1", special=False, phonetic=False)}}

\ExplSyntaxOn

\keys_define:nn { transcribe }
{
    ruby            .bool_set:N = \l_transcribe_ruby_bool,
    originalshow    .tl_set:N = \l_transcribe_original_tl,
    LVTfont         .tl_set:N = \l_transcribe_LVT_font_tl
}
\NewDocumentCommand\TranscriptionSetup{ m }
{
    \keys_set:nn { transcribe }{ #1 }
}

\NewDocumentEnvironment{transcription}{ s t| +b }
{
    \bool_if:NF \l_transcribe_ruby_bool
    {
        \str_if_eq:VnT \l_transcribe_original_tl { before }
        {
            #3 \par
        }
    }

    \seq_set_split:Nnn \l_tmpa_seq { \par }{ #3 }
    \IfBooleanTF{ #1 }
    {
        \seq_map_inline:Nn \l_tmpa_seq
        {
            \transcription_english_to_hangul:n { ##1 }
            \par
        }
    }{
        \IfBooleanTF{ #2 }
        {
            \seq_map_inline:Nn \l_tmpa_seq
            {
                \group_begin:
                \bool_if:NF \l_transcribe_ruby_bool { \l_transcribe_LVT_font_tl }
                \transcription_korean_to_LVT:n { ##1 }
                \par
                \group_end:
            }
        }{
            \seq_map_inline:Nn \l_tmpa_seq
            {
                \transcription_korean_to_roman:n { ##1 }
                \par
            }
        }
    }

    \bool_if:NF \l_transcribe_ruby_bool
    {
        \str_if_eq:VnT \l_transcribe_original_tl { after }
        {
            \par #3
        }
    }
}{}

\seq_new:N \l_tmpc_seq
\cs_new:Npn \transcription_korean_to_LVT:n #1
{
    \seq_set_split:Nnn \l_tmpb_seq { ~ }{ #1 }
    \bool_if:NTF \l_transcribe_ruby_bool
    {
        \seq_map_inline:Nn \l_tmpb_seq 
        {
            \ruby{##1}{\kortoLVT{##1}}\space
        }
    }{
        \seq_map_inline:Nn \l_tmpb_seq 
        {
            \kortoLVT{##1}\enspace
        }
    }
}

\cs_new:Npn \transcription_korean_to_roman:n #1
{
    \seq_set_split:Nnn \l_tmpb_seq { ~ }{ #1 }
    \bool_if:NTF \l_transcribe_ruby_bool
    {
        \seq_map_inline:Nn \l_tmpb_seq 
        {
            \ruby{##1}{\kortorom{##1}}\space
        }
    }{
        \seq_map_inline:Nn \l_tmpb_seq 
        {
            \kortorom{##1}\space
        }
    }
}

\cs_new:Npn \transcription_english_to_hangul:n #1
{
    \seq_set_split:Nnn \l_tmpb_seq { ~ }{ #1 }
    \bool_if:NTF \l_transcribe_ruby_bool
    {    
        \seq_map_inline:Nn \l_tmpb_seq 
        {
            \ruby{##1}{\engtohan{##1}}\space
        }
    }{
        \seq_map_inline:Nn \l_tmpb_seq 
        {
            \engtohan{##1}\space
        }
    }
}
\ExplSyntaxOff

\setlength\parindent{0pt}
\renewcommand\baselinestretch{2}

\begin{document}

\TranscriptionSetup{ruby=true}

\begin{transcription}|
우리나라 역사상 자유를 위한 가장 위대한 시위가 있었던 날로서 역사에 기록될 오늘 나는 여러분과 함께 하게 되어 행복합니다.
백년 전, 오늘 우리가 서있는 자리의 상징적 그림자의 주인공인, 한 위대한 미국인이, 노예해방선언문에 서명하였습니다.
\end{transcription}

\begin{transcription}
우리나라 역사상 자유를 위한 가장 위대한 시위가 있었던 날로서 역사에 기록될 오늘 나는 여러분과 함께 하게 되어 행복합니다.
백년 전, 오늘 우리가 서있는 자리의 상징적 그림자의 주인공인, 한 위대한 미국인이, 노예해방선언문에 서명하였습니다.
\end{transcription}

\begin{transcription}
구미 영동 백암 옥천 합덕 호법 월곶 벚꽃 한밭 구리 설악 칠곡 임실 울릉 대관령 백마 신문로 종로 왕십리 별내 신라 학여울 알약
해돋이 같이 굳히다 좋고 놓다 잡혀 낳지 독립문 속리산 집 짚 가곡
\end{transcription}

\begin{transcription}*
I am happy to join with you today in what will go down in history as the greatest demonstration for freedom in the history of our nation.
Five score years ago, a great American, in whose symbolic shadow we stand today, signed the Emancipation Proclamation.
\end{transcription}

\begin{transcription}*
coffee border monster bankrupt plural glacier courage echo lion client gap cat 
book apt setback act stamp cape nest part desk make apple mattress chipmunk sickness 
bulb land zigzag lobster kidnap signal mask jazz graph olive thrill bathe flash 
shrub shark shank fashion sheriff shopping shoe shim mirage vision keats odds switch 
bridge Pittsburgh hitchhike chart virgin steam corn ring lamp hint ink hanging longing
hotel pulp word want woe wander wag west witch wool time house skate oil boat tower 
swing twist penguin whistle quarter yard yank yearn yellow yawn you year Indian 
battalion union
\end{transcription}

\TranscriptionSetup{ruby=false, LVTfont=\scriptsize, originalshow=after}

\begin{transcription}
 유구한 역사와 전통에 빛나는 우리 대한국민은 3·1 운동으로 건립된 대한민국임시정부의 법통과 불의에 항거한 4·19 민주이념을 계승하고, 조국의 민주개혁과 평화적 통일의 사명에 입각하여 정의·인도와 동포애로써 민족의 단결을 공고히 하고, 모든 사회적 폐습과 불의를 타파하며, 자율과 조화를 바탕으로 자유민주적 기본질서를 더욱 확고히 하여 정치·경제·사회·문화의 모든 영역에 있어서 각인의 기회를 균등히 하고, 능력을 최고도로 발휘하게 하며, 자유와 권리에 따르는 책임과 의무를 완수하게 하여, 안으로는 국민생활의 균등한 향상을 기하고 밖으로는 항구적인 세계평화와 인류공영에 이바지함으로써 우리들과 우리들의 자손의 안전과 자유와 행복을 영원히 확보할 것을 다짐하면서 1948년 7월 12일에 제정되고 8차에 걸쳐 개정된 헌법을 이제 국회의 의결을 거쳐 국민투표에 의하여 개정한다.
\end{transcription}

\begin{transcription}|
 유구한 역사와 전통에 빛나는 우리 대한국민은 3·1 운동으로 건립된 대한민국임시정부의 법통과 불의에 항거한 4·19 민주이념을 계승하고, 조국의 민주개혁과 평화적 통일의 사명에 입각하여 정의·인도와 동포애로써 민족의 단결을 공고히 하고, 모든 사회적 폐습과 불의를 타파하며, 자율과 조화를 바탕으로 자유민주적 기본질서를 더욱 확고히 하여 정치·경제·사회·문화의 모든 영역에 있어서 각인의 기회를 균등히 하고, 능력을 최고도로 발휘하게 하며, 자유와 권리에 따르는 책임과 의무를 완수하게 하여, 안으로는 국민생활의 균등한 향상을 기하고 밖으로는 항구적인 세계평화와 인류공영에 이바지함으로써 우리들과 우리들의 자손의 안전과 자유와 행복을 영원히 확보할 것을 다짐하면서 1948년 7월 12일에 제정되고 8차에 걸쳐 개정된 헌법을 이제 국회의 의결을 거쳐 국민투표에 의하여 개정한다.
\end{transcription}


\end{document}