\documentclass[10pt, openright, language=korean]{hzguide}

\definecolor{GolfzonBlue}{cmyk}{1,0.8,0,0.2}
\definecolor{GlofzonLightBlue}{cmyk}{0.9,0.55,0,0}

% \setmainfont{TeX Gyre Heros}
% \setmainhangulfont{NotoSansCJKkr}[Ligatures=TeX, Scale=MatchUppercase, UprightFont={*-Regular}, BoldFont={*-Bold}]
%\setsanshangulfont{GOLFZON}[Ligatures=TeX, UprightFont={*R}, BoldFont={*B}]

\CoverSetup{
    AfterFront=\cleartorecto,
    BeforeBack={\cleartorecto\cleartoverso}
}

\LayoutSetup{
    %showtrims,
    %showlayout,
    paper=B5, 
    ulmargin=21mm, % textheight: 250 - 42 = 208 mm (= 130 * 1.6)
    lrmargin=23mm, % textwidth: 130 mm (= 176 - 46)
    column=one
}

\HeadingSetup{
    chapterstyle=tandh,
    chaptercolor = GolfzonBlue,
    sectionstyle = sparse,
    sectioncolor = GolfzonBlue,    
    %sectioncolor = GolfzonLightBlue
    sectionsize = \LARGE,    
    subsectionsize = \large
}
% \setlength{\beforechapskip}{1\onelineskip}
% \setlength{\afterchapskip}{2\onelineskip}

\renewcommand*{\chapnumfont}{\color{\chaptercolor}\normalfont\Huge\bfseries}

\setsecnumdepth{part}
%\SectionNewpageOn
\DecolorHyperlinks[GolfzonBlue]

\ExplSyntaxOn
\RenewDocumentEnvironment { Caution } { s t{|} }
{
    \IfBooleanTF{#1}
    {
        \IfBooleanTF{#2}
        { \begin{admonition}*|{\l_admonition_caution} }
        { \begin{admonition}*{\l_admonition_caution} }
    }{
    \IfBooleanTF{#2}
        { \begin{admonition}|{\l_admonition_caution} }
        { \begin{admonition}{\l_admonition_caution} }
    }
}
{
    \end{admonition}
}
\AdmonitionSetup{font=\small}

\IllustImageSetup{
    firstskip=.5\baselineskip,
    voffset=-.25\baselineskip,
    enum={voffset=-.5\baselineskip}
}
%\imagescale{0.95}
%\ShowImageFilename*

\CalloutSetup{
    rule=false,
    label=\cirnum{\calloutno}\enspace,
    labelenum=\cirnum{\theenumi},
}

\TermsSetup{    
    star={delimiter=\hfill},
    vbar={font={}, index=false},
    %enummarker={\cirnum{\int_use:N \l_terms_int}\hskip\l_terms_markersep} 
}
\ExplSyntaxOff

\uisetup{
    font=\sffamily\bfseries,
    color=darkgray
}

\ActionSetup {
    delimiter = :,
    inline = false
}

%\IndexingEnable*
\IfIndexing{
    \newcommand\PrintIndex{\backmatter\printindex}
}[
    \newcommand\PrintIndex{}
]

% for \path{directory path}
\def\flushleft{\setlength\topsep{0pt}\trivlist \raggedright\item\relax}

\CoverSetup{    
    FrontLogoImage = logo_TwovisionPlus,
    BackLogoImage = logo_Golfzon,
    ProductImage = {},
    title = {투비전 플러스},
    subtitle = {실내 골프 시뮬레이터},
    DocumentType = {운영 설명서},
    PubYear = 2018,
    revision = {Rev. 1},
    note = {필요할 때 언제든지 볼 수 있도록 이 매뉴얼을 가까운 곳에 보관하십시오.},
    manufacturer = ㈜골프존,
    address = {대전광역시 유성구 엑스포로97번길 40}
}

\begin{document}

\frontmatter* 

\FrontCover

\tableofcontents*

\chapter{일러두기}

이 문서는 ㈜골프존의 투비전 플러스를 사용하는 방법에 대해 설명합니다. 이 문서의 내용 중 제품 사양과 일부 기능은 사전 예고 없이 바뀔 수 있습니다. 이 문서에 사용된 삽화들은 설명을 돕기 위한 것으로, 설치 옵션에 따라 실제와 다소 다를 수 있습니다. 스크린 이미지들은 소프트웨어 업데이트에 따라 달라질 수 있습니다.

\section{지식 재산권}

Copyright 2018 ㈜골프존

삽화를 포함하여 이 문서에 포함된 모든 내용은 ㈜골프존의 소유입니다. ㈜골프존의 동의 없이 이 문서의 전부 또는 일부를 어떤 방법으로도 복사하거나 재배포하는 것은 허용되지 않습니다.

\section{표기 규약}

\begin{Warning}
"경고"와 함께 주어진 지시를 따르지 않으면 부상을 입을 수 있습니다.
\end{Warning}

\begin{Caution}
"주의"와 함께 주어진 지시를 따르지 않으면 장비가 손상될 수 있습니다.
\end{Caution}

\begin{Note}
"참고"와 함께 유용한 정보가 추가로 제시됩니다.
\end{Note}

\mainmatter*

\chapter{소개}

투비전 플러스는 실제와 동일한 골프 환경을 구현하는 골프 시뮬레이터입니다.
사용자들은 실제와 동일한, 다양한 골프 코스뿐만 아니라 드라이빙 레인지도 한 자리에서 경험할 수 있습니다.

\section{투비전 플러스 구성}

%투비전 플러스는 스크린과 프로젝터를 비롯하여 여러 장치들로 이루어져 있습니다.

\begin{Note}
이 구성도는 오른손잡이와 왼손잡이 모두가 이용할 수 있는 설비를 보여줍니다.
대부분의 경우에 오른손잡이를 위한 설비만 갖추어져 있습니다.
실제 구성은 설치 옵션에 따라 달라질 수 있습니다.
\end{Note}

\image{diagram_TwovisionPlus}

\begin{callout}
\item 바닥 프로젝터
\item 투비전 센서 
\item 정면 프로젝터
\item 키오스크 
\item 5.1 채널 스피커 
\item 키패드
\item 듀얼 플레이트
\item 바닥 스크린
\item 정면 스크린
\end{callout}

\begin{Note}
프로젝터를 비롯하여 일부 장치들은 기본 품목과 고급 품목으로 나뉩니다.
고급 품목은 선택 사양이자 권장 사양입니다.
이 문서는 권장 사양으로 이루어진 경우를 전제로 합니다.
\end{Note}

\subsection{키오스크}

키오스크를 구성하는 장치들은 다음과 같습니다.

\image{diagram_kiosk}

\begin{callout}*
\item 터치 모니터: 터치 모니터를 사용하여 투비전 플러스 프로그램을 조작하십시오.
\item 전원 스위치: 전원 스위치를 눌러서 키오스크를 켜십시오.
\item 카드 리더기: 골프존 앱이나 투비전 앱이 설치된 스마트폰, 또는 회원 카드를 카드 리더기에 접촉하여 투비전 플러스에 빠르게 로그인하십시오.
\item 나스모 카메라: 플레이어가 볼을 치는 순간을 카메라가 녹화합니다. 플레이어는 녹화된 영상을 재생하여 볼 수 있습니다.
\item 투비전 센서: 센서가 볼이 골프 클럽에 맞는 순간을 플레이어 앞에서 센싱합니다.
\end{callout}

\begin{ImageTable}
\lineimg{comp_touch_monitor} & \lineimg{comp_camera} & \lineimg{comp_sensor_kiosk_rotated} \\
터치 모니터 & 나스모 카메라 & 투비전 센서 \\ 
\end{ImageTable}

\newpage
\subsection{투비전 센서}

천장과 키오스크에 설치된 센서가 볼이 골프 클럽에 맞는 순간을 플레이어의 위와 앞에서 센싱합니다. 

\begin{ImageTable}*
\lineimg{comp_sensor_ceiling} & \lineimg{comp_sensor_kiosk_rotated}  \\
천장에 설치된 센서 & 키오스크에 설치된 센서 \\
\end{ImageTable}

\subsection{듀얼 플레이트}
\label{sec:swing_plate}

사용자들은 듀얼 플레이트 위에서 스크린을 향해 볼을 칩니다.
듀얼 플레이트는 다음과 같이 여러 부분으로 나뉩니다.

\image{diagram_swing_plate}

\begin{callout}*
\item 페어웨이 매트: 볼이 페어웨이나 그린에 위치할 때 페어웨이 매트에 볼을 놓고 치십시오.
\item 투어 러프 매트: 러프에서 샷을 해야 할 때 러프 매트에 볼을 놓고 치십시오.
\item 투어 벙커 매트: 벙커에서 샷을 해야 할 때 벙커 매트에 볼을 놓고 치십시오.
\item 오토티업기: 고무 티가 볼을 이고 올라옵니다. 볼을 매트에 내려놓으면 고무 티가 내려갑니다.
\item 타석 매트: 타석 매트에 올라서서 볼을 치십시오.
\item 퍼팅 가이드: \pageref{sec:putting_guide} 쪽 \titleref{sec:putting_guide}\를 보십시오.
\end{callout}

\begin{ImageTable}
\lineimg{comp_mat_fairway} & \lineimg{comp_mat_rough} & \lineimg{comp_mat_bunker} \\
페어웨이 매트 & 투어 러프 매트 & 투어 벙커 매트 \\
\end{ImageTable}

\subsection{키패드}

듀얼 플레이트 옆에 키패드가 놓여져 있습니다.
키패드를 사용하여 타격 방향을 조정하거나 클럽을 다른 것으로 바꿀 수 있습니다.

\image{diagram_control_pad}

\begin{callout}*
\item 퍼팅이 가능할 때 격자 버튼을 누르면 화면에 격자가 표시되거나 사라집니다. 게임 옵션에 따라 이 기능을 사용하지 못할 수 있습니다.
\item 티샷을 할 때 위쪽 버튼을 누를 때마다 티가 5 mm 올라갑니다. 티샷이 아닌 경우에는 이 버튼을 이용하여 더 긴 클럽으로 바꿀 수 있습니다.
\item 티샷을 할 때 아래쪽 버튼을 누를 때마다 티가 5 mm 내려갑니다. 티샷이 아닌 경우에는 이 버튼을 이용하여 더 짧은 클럽으로 바꿀 수 있습니다.
\item 왼쪽 버튼을 누를 때마다 목표 지점이 조금씩 왼쪽으로 이동합니다. 이동 거리는 상황에 따라 달라집니다.
\item 오른쪽 버튼을 누를 때마다 목표 지점이 조금씩 오른쪽으로 이동합니다.
\item 티업 버튼을 누를 때마다 오토티업기에 의해 볼이 하나씩 공급됩니다.
\end{callout}

\subsection{프로젝터}

정면 프로젝터가 코스 영상을 정면 스크린에 영사합니다.
볼이 홀로부터 4 미터 이내의 거리에 도달했을 때, 바닥 프로젝터가 그린 영상을 바닥 스크린에 영사합니다.

\begin{ImageTable}*
\lineimg{comp_projector_wall_low} & \lineimg{comp_projector_floor_low}  \\
정면 프로젝터 & 바닥 프로젝터 \\
\end{ImageTable}

% \begin{ImageTable}*
% \lineimg{comp_projector_wall_low} & \lineimg{comp_projector_floor_low}  \\
% 정면 프로젝터 (기본) & 바닥 프로젝터 (기본) \\
% \lineimg{comp_projector_wall} & \lineimg{comp_projector_floor}  \\
% 프리미엄 정면 프로젝터 (선택) & 프리미엄 바닥 프로젝터 (선택) \\
% \end{ImageTable}

\newpage
\subsection{스크린}

매장의 크기에 맞춰 스크린이 제공됩니다. 

\begin{ImageTable}*
\lineimg{comp_screen_wall} & \lineimg{comp_screen_floor}  \\
정면 스크린 & 바닥 스크린 \\
\end{ImageTable}

\begin{Caution}
깨진 볼을 사용하지 마십시오. 스크린이 손상됩니다.
\end{Caution}

\subsection{5.1 채널 스피커}

천장과 벽과 바닥에 설치된 6 개의 개별 스피커가 입체 음향을 구현하여 게임의 생동감을 극대화합니다.

\image{comp_speakers}

\newpage
\section{투비전 플러스의 주요 특징}

투비전 플러스에 추가된 다음과 같은 새로운 기능과 특징이 기존 제품보다 더욱 현실감 높은 골프 게임을 구현합니다.

\begin{itemize}
\item 움직이는 듀얼 플레이트가 다양한 지형의 언듀레이션을 구현하여 실제 필드에 있는 듯한 느낌을 제공합니다.
\item 퍼팅 가이드가 녹색과 적색의 불빛으로 홀 방향과 퍼팅 방향을 안내하여 플레이어의 퍼팅을 돕습니다.
\item 5.1 채널 스피커와 HRTF 적용으로 더욱 생동감있는 필드 느낌을 구현했습니다.
\item 새로운 프로젝터가 기존 제품에 비해 더욱 선명한 영상을 제공합니다.
\item 투어 러프 매트, 투어 벙커 매트 등 필드 환경을 최대한 구현한 매트를 통해 실감나는 게임을 즐길 수 있습니다.
%\item 정면 스크린에 바닥 스크린을 더해 실제 필드와 유사한 환경을 제공하여 게임의 몰입도가 향상되었습니다.
%\item 실제 골프장들과 동일한 코스들을 경험할 수 있습니다.    
\item 투비전 플러스는 볼 타격뿐만 아니라 디봇 영역까지 센싱하여 보다 정확한 타격 피드백을 제공합니다.
\item 더욱 빠르고 흥미진진한 게임 전개를 위해 익스트림 골프라는 새로운 경기 규칙이 추가되었습니다.
\item 기존 제품이 제공하는 게임 모드들에 더하여, 미니 라운드를 비롯한 새로운 게임 모드들이 투비전 플러스에 추가되었습니다.
\item 미션 시스템이 추가되었습니다. 미션들을 하나씩 달성하면서 골프 실력이 크게 향상될 것입니다.    
\end{itemize}

각 기능과 특징에 대한 자세한 내용은 다음 절들을 보십시오.

\newpage
\subsection{듀얼 플레이트}

사용자들은 듀얼 플레이트라는 구조물 위에서 스크린을 향해 볼을 칩니다. 듀얼 플레이트는 단순한 플랫폼이 아니고, 실제 지면과 흡사한 비탈을 만들어내는 구동 장치입니다.

하나의 판으로 단조로운 경사를 만들어내는 기존 제품과 달리, 투비전 플러스의 듀얼 플레이트는 두 부위로 이루어져 있습니다. 플레이어가 위치하는 부위와 볼이 놓이는 부위가 독립적으로 상승하거나 하강하여 아래 예들을 포함하여 총 14 가지 형태의 언듀레이션을 만들어냅니다.

\image{diagram_swing_plate_tilt}

\newpage
\subsection{퍼팅 가이드}
\label{sec:putting_guide}

볼이 홀로부터 4 미터 이내의 거리에 도달했을 때, 바닥 스크린에 그린 영상이 나타나고 듀얼 플레이트의 전면에 부착된 퍼팅 가이드가 작동합니다.

\image{diagram_putting_guide}

퍼팅 가이드에는 여러 개의 지시등이 달려 있습니다.
홀의 위치에 따라 두 개의 지시등이 켜져서 플레이어의 퍼팅을 돕습니다.
녹색으로 켜진 지시등이 홀 방향을 가리키고, 적색으로 켜진 지시등이 퍼팅 방향을 가리킵니다.

퍼팅 가이드가 작동하지 않도록 설정할 수 있습니다. 
\pageref{sec:admin_setting_default} 쪽 \titleref{sec:admin_setting_default}\를 보십시오.

\subsection{5.1 채널 스피커}

\image{diagram_speakers}

5.1 채널 스피커와 HRTF 기술이 결합하여 만들어내는 입체 음향이 실제 필드에 있는 듯한 느낌을 더욱 배가합니다.

\subsection{레이저 프로젝터}

새로 도입된 고사양 프로젝터는 더 선명한 영상을 제공합니다.

\begin{ImageTable}*
\lineimg{comp_projector_wall} & \lineimg{comp_projector_floor}  \\
정면 프로젝터 & 바닥 프로젝터 \\
\end{ImageTable}

\subsection{투어 러프 매트}

\image{diagram_mat_rough}

같은 길이의 잔디로 이루어진 기존 제품과 달리, 투비전 플러스의 러프 매트는 표면의 절반에 긴 잔디로, 나머지 면에 짧은 잔디로 이루어져 있습니다.
긴 잔디가 촘촘한 러프의 느낌을, 짧은 잔디가 성긴 러프의 느낌을 제공하여, 사용자들은 더욱 실감나는 게임을 즐길 수 있습니다.

\subsection{투어 벙커 매트}

\image{diagram_mat_bunker}

같은 길이의 올로 이루어진 기존 제품과 달리, 투비전 플러스의 벙커 매트는 번갈아 짜인 긴 올과 짧은 올로 이루어져 있습니다.
올 사이에 볼을 얕이 또는 깊이 둠으로써 사용자들은 얕은 벙커와 깊은 벙커의 느낌을 살릴 수 있습니다.

\subsection{디봇 센싱}

\image{screen_divot}

투비전 플러스는 클럽 헤드가 볼에 부딪치는 각도뿐만 아니라 클럽 헤드가 얼마나 길고 깊게 매트 표면을 긁고 지나가는지도 감지합니다.
마치 잔디가 뜯겨나가는 것과 같은 정도의 타격에 대해 실제에 가까운 결과를 투비전 플러스가 보여줍니다.

\subsection{익스트림 골프}

익스트림 골프라는 새로운 경기 규칙이 투비전 플러스에 추가되었습니다. 이 규칙을 이용하여 게임을 더 빠르고 박진감 있게 진행할 수 있습니다.

\image{screen_extreme_rule}

다음과 같은 규칙들이 적용됩니다.

\begin{itemize}
\item 플레이어들은 자기 차례에서 티샷부터 볼이 홀에 들어갈 때까지 연속하여 볼을 칩니다.
\item 플레이어들은 매 샷마다 20 초 안에 볼을 쳐야 합니다. 제한 시간은 변경할 수 있습니다.
\item 제한 시간을 넘기면 1 타의 페널티가 타수에 더해집니다.
\item 누적 타수와 사용 시간을 합하여, 가장 낮은 점수를 올린 플레이어가 승자가 됩니다.
예를 들어, 20 분 34 초 동안 72 타를 기록한 플레이어의 점수는 다음과 같습니다.
\begin{reference}
72 + 20.34 = 92.34
\end{reference}
\end{itemize}

\subsection{미니 라운드}

실제 필드에서 이루어지는 경기 방식들뿐만 아니라 가상 공간과 온라인 공간에서 구현되는 경기 방식들을 투비전 플러스에서 즐길 수 있습니다. 미니 라운드 모드는 투비전 플러스에 새로 추가된 게임 모드들 가운데 가장 흥미진진하고 이채로운 경기 방식입니다.

\image{screen_mini_round}

미니 라운드 모드에서는 연습장 같은 코스가 제공됩니다. 실제보다 단순화된 골프 코스가 제공되어, 플레이어들은 짧은 시간 동안 코스의 특징을 파악하고 경기 전략을 구상할 수 있습니다. 실제 골프장을 방문하기 전에 미니 라운드 모드로 연습해 보십시오. 우승 확률이 높아질 것입니다.

미니 라운드 게임의 특징은 다음과 같습니다.

\begin{itemize}
\item 개인 단위 시합만 가능합니다.
\item 두 명 이상이 동일한 타수로 그린에 도달하였다면, 홀에 가장 가까운 볼의 소유자에게 가산점이 주어집니다.
\item 익스트림 골프를 사용할 수 없습니다.
\item 멀리건을 제한없이 사용할 수 있습니다.
\end{itemize}

\subsection{미션 시스템}

\image{screen_user_missions}

골프존 사용자들을 위한 다양한 미션들이 있습니다.
예를 들어, 멀리건을 사용하지 않고 페널티 없이 라운드를 마치면 \ui{공 하나면 충분해} 미션이 달성됩니다.
미션에 성공하면 티를 단위로 하는 일정 포인트가 지급됩니다.
사용자들은 자신이 획득한 티 포인트를 골프 장갑 같은 물건으로 교환할 수 있습니다.
골프존 모바일 앱에서 각 미션에 대한 자세한 내용과 보상, 그리고 사용자의 미션 달성률을 확인할 수 있습니다.



% \section{골프존 계정}

% 골프존 계정이 있으면 투비전 플러스를 더 쉽게 이용할 수 있습니다.
% 골프존 웹사이트 \url{http://www.golfzon.com}를 방문하여 계정을 만드십시오.
% 또한 골프존 가맹점에서 회원 카드를 만들 수 있습니다.

% \newpage
% \section{모바일 앱}

% 골프존 앱과 투비전 앱을 이용하여 투비전 플러스를 더 편리하게 즐길 수 있습니다.

% \subsection{골프존 앱}

% 다음 요건들을 충족하는 스마트폰에 골프존 앱을 설치하십시오.

% \begin{itemize}
% \item 안드로이드 4.0.3 이상 또는 iOS 9.0 이상
% \item NFC
% \end{itemize}

% \image*{app_golfzon}

% 골프존 앱을 이용하여 다음과 같은 것들을 할 수 있습니다.

% \begin{itemize}
% \item 투비전 플러스에 간편하고 빠르게 로그인
% \item 닉네임을 비롯한 플레이어 옵션 설정
% \item 게임 아이템 구매
% \item 골프 동호회 개설
% \item 동호회 게임 개최
% \item 스윙 영상 공유
% \end{itemize}

% \subsection{투비전 앱}

% 다음 요건들을 충족하는 스마트폰에 투비전 앱을 설치하십시오.

% \begin{itemize}
% \item 안드로이드 4.0.3 이상
% \item 블루투스 4.0 이상
% \end{itemize}

% \image{app_twovision}

% 투비전 앱을 이용하여 다음과 같은 것들을 할 수 있습니다.

% \begin{itemize}
% \item 투비전 플러스에 간편하고 빠르게 로그인
% \item 홀 번호와 기준 타수를 비롯한, 현재 진행 중인 게임에 대한 정보 확인
% \item 다양한 이모티콘 스티커를 스크린에 표시
% \item 홀 공략에 도움이 되는 상세한 정보 파악
% \item 키패드와 터치 모니터를 대신하여, 게임 기능을 원격으로 제어 
% \end{itemize}

\chapter{게임 준비}

\section{투비전 플러스 시작하기}

다음 절차를 따라 투비전 플러스를 시작하십시오.

\begin{enumerate}
\item 키오스크를 비롯하여 각 장치를 켜십시오.
\item 키오스크 화면에서 \lineimg{icon_Twovision_launcher} 투비전 플러스 아이콘을 터치하십시오. 대화상자가 나타나서 진행되는 소프트웨어 업데이트를 보여줍니다.
\item 소프트웨어 업데이트가 완료되면 \ui{START}를 터치하십시오.
\item 나타난 투비전 플러스의 시작 화면에서 \ui{시작}을 터치하십시오.
\end{enumerate}

\image{screen_Twovision_home}

게임을 준비하기 위한 일반적인 절차는 다음과 같습니다.

\begin{enumerate}
\item 플레이어를 추가하십시오.
\item 게임 모드를 선택하십시오.
\item 골프 코스를 선택하십시오. 선택된 게임 모드에 따라 이 단계가 생략될 수 있습니다.
\item 득점 방식을 비롯하여 게임 조건들을 설정하십시오. 선택된 게임 모드에 따라 이 단계가 생략되거나 설정할 수 있는 옵션들이 달라집니다.
\end{enumerate}

\ui{이전} 버튼과 \ui{다음} 버튼, 또는 \lineimg{button_left} 버튼과 \lineimg{button_right} 버튼을 이용하여 이전 단계 또는 다음 단계로 이동하십시오.

\newpage
\section{1 단계: 플레이어 추가하기}

한 게임에 1 명부터 6 명까지 참여할 수 있습니다.

\image{screen_player_members}

다음 절차를 따라 플레이어들을 추가하고 플레이어 옵션들을 설정하십시오.
각 단계의 자세한 방법에 대해서는 다음 절들을 보십시오.

\begin{enumerate}
\item 골프존 계정으로 로그인하십시오.
\item 각 플레이어에 대해 플레이어 옵션들을 설정하십시오.
\item 각 플레이어가 보유한 게임 아이템들 가운데 무엇을 사용할지 설정하십시오.
\item \ui{다음}을 터치하여 다음 단계로 이동하십시오.
\end{enumerate}

\subsection{플레이어 추가하기}

골프존 회원이 투비전 플러스에 로그인할 수 있는 여러 편리한 방법들이 제공됩니다.
다음 방법들 가운데 하나를 이용하여 로그인하십시오.

\begin{itemize}
\item 키오스크 화면에서 \lineimg{button_add_card} 버튼을 터치하고 회원 카드를 키오스크의 카드 리더기에 대십시오.
\item 골프존 앱이나 투비전 앱이 스마트폰에 설치되어 있다면, 다음 방법들 가운데 하나를 시도하십시오.
    \begin{itemize}
    \item 스마트폰의 NFC 기능이 활성화되어 있는지 확인하고, 스마트폰을 키오스크의 카드 리더기에 대십시오. 골프존 앱이 설치되어 있어야 합니다.
    \item 투비전 앱을 열고 \ui{지금 연결하기}를 터치하십시오. 투비전 앱이 블루투스를 통해 키오스크와 연결될 때까지 기다리십시오.
    \item 키오스크 화면의 왼쪽 상단에 표시된 번호를 골프존 앱이나 투비전 앱에 입력하십시오.
    \listimg{screen_terminal_id}
    \end{itemize}
\item 키오스크 화면에서 \ui{로그인}을 터치하고 아이디와 패스워드를 입력하십시오.
\end{itemize}


\action{골프존 계정이 없는 사용자를 플레이어로 추가하기}
\ui{게스트 등록}을 터치하십시오. 임의의 닉네임이 부여됩니다.
닉네임 옆에 \lineimg{button_edit} 버튼을 터치하고 닉네임을 변경하십시오.

\action{플레이어를 목록에서 제거하기}
플레이어를 선택하고 닉네임 옆에 \lineimg{button_remove} 버튼을 터치하십시오.

\subsection{플레이어 옵션 설정하기}

각 플레이어에 대해 다음 옵션들을 설정하십시오.

\begin{UI}
\item[난이도] 플레이어 수준을 설정하십시오. 
\item[티 위치] 티잉 그라운드에서 티를 어느 쪽에 가깝게 둘지 선택하십시오.
\item[티 높이] 티의 높이를 선택하십시오.
\item[타석 위치] 플레이어가 왼손잡이인지 오른손잡이인지 설정하십시오.
\end{UI}

\ui{전체 설정} 옵션을 이용하여 동일한 난이도와 티 위치를 모든 플레이어에 대해 동시에 설정할 수 있습니다. 

\subsection{게임 아이템 선택하기}

게임에서 사용할 수 있는, 가상 캐디나 볼과 같은 게임 아이템을 골프존 앱이나 웹사이트에서 구매할 수 있습니다.
예를 들어, 리본 볼꼬리 아이템을 구매하면 볼이 날아갈 때 볼의 궤적이 노란 리본으로 장식됩니다.

\image{screen_user_items}

다음 절차를 따라 게임에서 사용할 게임 아이템들을 선택하십시오.

\begin{enumerate}
\item 플레이어를 선택하고 \ui{나의 플러스}를 터치하십시오.
\item 나타난 대화상자에서 사용할 아이템을 종류별로 선택하십시오.
\item 마치려면 \ui{확인}을 터치하십시오.
\end{enumerate}

\subsection{플레이어 기록 보기}

라운드 수를 비롯하여, 한 플레이어가 이제까지 쌓은 게임 기록을 볼 수 있습니다.

\image{screen_user_record}

다음 절차를 따라 플레이어의 게임 기록을 조회하십시오.

\begin{enumerate}
\item 플레이어를 선택하고 닉네임 옆에 \lineimg{button_flag} 버튼을 터치하십시오.
대화상자가 나타나서 선택된 플레이어의 게임 기록을 보여줍니다.
\item 대화상자를 닫으려면 \ui{닫기}를 터치하십시오.
\end{enumerate}

\newpage
\section{2 단계: 게임 모드 선택하기}
\label{sec:game_modes}

스트로크를 비롯한 다양한 게임 모드들이 제공됩니다.

\image{screen_game_modes}


원하는 게임 모드를 선택하고 \ui{다음}을 터치하십시오.
각 게임 모드에 대한 자세한 내용은 다음 절들을 보십시오.

\subsection{스트로크}

스트로크 모드가 가장 많이 이용되는 게임 모드입니다.
이 모드에서 게임 조건들을 자유롭게 설정할 수 있습니다.

\subsection{대회}

개인이나 단체가 개설한 여러 종류의 대회에 참여할 수 있습니다.
개설자들이 정한 조건과 규칙에 따라 수 일 또는 수십 일 동안 대회들이 진행됩니다.
개설자들은 다음과 같은 것들을 정할 수 있습니다.

\begin{itemize}*
\item 대회 기간
\item 참가 자격
\item 골프 코스
\item 멀리건을 비롯한 경기 조건
\end{itemize}

대회를 개설하려면 골프존 웹사이트를 방문하십시오.

\begin{Note}
대회 개설자의 설정에 따라 가상 캐디와 같은 게임 아이템의 사용이 허용되지 않을 수 있습니다.
\end{Note}

\subsection{배틀존}

배틀존 모드의 게임에서 매 홀마다 일정 포인트가 걸립니다. 아마추어 리그를 비롯하여 세 가지 배틀존 리그가 있으며, 리그에 따라 홀에 걸리는 포인트의 수가 달라집니다.
홀의 승자는 그 홀에 걸린 만큼 포인트를 얻고, 패자는 그만큼 포인트를 잃습니다.
한 홀이 무승부로 끝나면 그 다음 홀에서 포인트 수가 두 배로 늘어납니다.
프로 리그나 투어 리그의 배틀존 게임에 참여하려면 그 리그가 요구하는 만큼의 포인트를 보유해야 합니다.

두 가지 배틀존 모드가 있습니다.

\begin{UI}
\item[배틀 플레이] 이 모드에서 진행된 모든 게임들이 골프존 서버에 저장됩니다.
저장된 게임들 가운데 하나를 선택하십시오.
선택된 게임이 재생되면서, 마치 원격지에 있는 사람들과 실시간으로 경기하는 것처럼 게임을 즐길 수 있습니다.
\item[조인 플레이] 원격지에 있는 사람들과 실시간으로 게임을 할 수 있습니다.
게임 방을 만들고 다른 곳에 있는 사람들을 초대하거나, 게임 방을 찾아 들어가십시오.
\end{UI}

\subsection{골친 라운드}

골친 라운드 모드나 스트로크 모드에서 진행된 게임은 골프존 서버에 저장되고 지인들에게 공유될 수 있습니다.
공유된 게임들 가운데 하나를 선택하십시오.
마치 지인들과 실시간으로 경기하는 것처럼 게임을 즐길 수 있습니다.

\subsection{클럽 대전}

한 동호회의 회원들끼리 또는 다른 동호회를 상대로 실시간으로 게임을 할 수 있습니다.
개설된 동호회 게임들 가운데 하나를 선택하십시오.

\begin{Note}
골프존 앱을 이용하여 골프 동호회를 만들거나 가입하십시오. 동호회 대항 게임도 골프존 앱을 이용하여 개설할 수 있습니다.
\end{Note}

\subsection{미니 라운드}

이 모드에서는 연습장 같은 코스가 제공됩니다. 실제보다 단순화된 골프 코스가 제공되어, 플레이어들은 짧은 시간 동안 코스의 특징을 파악하고 경기 전략을 구상할 수 있습니다.

이 모드에서는 개인 단위 시합만 가능합니다. 두 명 이상이 동일한 타수로 그린에 도달하였다면, 홀에 가장 가까운 볼의 소유자에게 가산점이 주어집니다. 익스트림 골프를 사용할 수 없다는 점과 제한없이 멀리건을 사용할 수 있다는 점이 미니 라운드 게임의 또 다른 특징입니다.

\subsection{포섬 스트로크}

이 모드에서는 네 명이 두 팀으로 나뉘어 대결합니다. 한 팀에 속한 두 사람이 하나의 공을 함께 사용합니다.

\subsection{매치}

매치 모드에서는 두 사람이 일대일로 대결합니다. 홀마다 승자를 가리고 더 많은 홀을 차지한 사람이 라운드 우승자가 됩니다.

\subsection{칩 앤 펏}

이 모드에서는 모든 홀의 기준 타수가 3입니다. 이 모드에서 컨시드 기능은 허용되지 않습니다.

\newpage
\section{3 단계: 골프 코스 선택하기}

\begin{Note}
선택된 게임 모드에 따라 이 단계가 생략될 수 있습니다.
\end{Note}

그린 난이도가 높을수록 그린의 언듀레이션이 심하고, 코스 난이도가 높을수록 벙커를 비롯한 장애물들이 많습니다.
아래에 제시된 방법들을 이용하여 원하는 골프 코스를 찾아 선택하고 \ui{다음}을 터치하십시오.

\image{screen_course_select}

\action{코스 목록 정렬하기}
\ui{그린 난이도}와 \ui{코스 난이도}를 터치하십시오.
코스 난이도와 그린 난이도를 기준으로 코스 목록이 오름차순으로 또는 내림차순으로 정렬됩니다.

\action{코스 찾기}
\ui{CC 검색}을 터치하고 골프장 이름을 입력하십시오.

\action{사용자들이 선호하는 코스 보기}
\ui{인기 CC}를 터치하십시오.

\action{최근에 추가되거나 변경된 코스 보기}
\ui{신규/이벤트 CC}를 터치하십시오.

\action{임의의 코스를 선택하기}
\ui{랜덤 CC}를 터치하십시오.

\newpage
\section{4 단계: 게임 조건 설정하기}
\label{sec:game_options}

\begin{Note}
선택된 게임 모드에 따라 이 단계가 생략될 수 있습니다.
\end{Note}

스트로크 게임 모드에서 다양한 옵션들을 이용하여 자유롭게 게임 조건들을 설정할 수 있습니다.
대회 모드나 그와 유사한 게임 모드에서는 게임 개설자에 의해 게임 조건들이 미리 설정되어 있습니다.
다른 게임 모드에서는 설정할 수 있는 옵션들이 제한됩니다.

스트로크 모드가 선택된 경우에 다음과 같은 대화상자가 나타납니다.

\image{screen_stroke_mode}

각 스트로크 모드의 특징은 다음과 같습니다.

\begin{UI}
\item[일반 모드] 이 모드에서는 모든 게임 옵션들을 자유롭게 설정할 수 있으며, 게임 중에 모든 기능들을 사용할 수 있습니다. 코스 지도에 방향 안내선이 표시됩니다.
\item[세미투어 모드] 이 모드에서도 모든 게임 옵션들을 자유롭게 설정할 수 있습니다. 코스 지도에 방향 안내선이 표시되지 않고, 대신 일정 간격으로 거리를 나타내는 동심원이 표시됩니다.
\item[투어 모드] 이 모드에서는 일부 게임 옵션들이 어려운 조건으로 고정되어 있고, 격자 표시를 비롯한 일부 기능들을 사용할 수 없습니다. 코스 지도에 방향 안내선이 표시되지 않고, 대신 일정 간격으로 거리를 나타내는 동심원이 표시됩니다.
\end{UI}

\begin{ImageTable}
\lineimg{screen_stroke_normal} & \lineimg{screen_stroke_semitour}  &  \lineimg{screen_stroke_tour} \\
일반 모드 & 세미 투어 모드 & 투어 모드
\end{ImageTable}

각 모드의 차이는 다음 표와 같습니다. 볼이 러프나 벙커에 떨어진 상황에서 러프 매트나 벙커 매트 대신 페어웨이 매트에 볼을 놓고 치면 볼 속도가 일정 비율로 줄어듭니다.

\begin{Table}
\begin{tabu}{XX[c]X[c]X[c]}
\hline
\TableHeadFont & 일반 모드 & 세미 투어 모드 & 투어 모드 \\
\hline
미니맵 & \foreign{◯} & \foreign{✗} & \foreign{✗} \\
야디지북 & \foreign{✗} & \foreign{◯} & \foreign{◯} \\
그린 격자 & \foreign{◯} & \foreign{◯} & \foreign{✗} \\
러프 & 0\% & 10\% & 20\% \\
필드에 있는 벙커 & 0\% & 20\% & 40\% \\
그린 앞 벙커 & 0\% & 20\% & 40\% \\
\hline
\end{tabu}
\end{Table}

이용할 스트로크 모드를 선택하고 \ui{확인}을 터치하십시오.

\image{screen_game_options}

다음 절차를 따라 게임 조건들을 설정하십시오.

\begin{enumerate}
\item 아래에 제시된 설명을 참조하여 옵션들을 적절하게 설정하십시오. \ui{쉬움}, \ui{보통}, 또는 \ui{어려움} 버튼을 이용하여 모든 항목들을 한번에 설정할 수 있습니다.
\item \ui{기타 설정}을 터치하고, 다른 옵션들을 적절하게 설정하십시오. 
\item 마치려면 \ui{라운드 시작}을 터치하십시오. 게임이 시작됩니다.
\end{enumerate}

\begin{UI}
\item[핀 위치] 그린에서 홀이 위치할 자리를 지정하십시오.
\item[그린 위치] 두 곳의 그린 중 어느 쪽을 사용할지 선택하십시오.
\item[그린 상태] 그린의 빠르기를 선택하십시오.
\item[컨시드] 그린에서 홀로부터 특정한 거리 안에 볼이 이르렀을 때 볼이 홀에 들어간 것으로 인정하는 것을 컨시드라고 합니다. 컨시드 거리를 지정하십시오.
\item[퍼팅 격자] 퍼팅이 가능할 때 화면에 격자가 표시됩니다. 격자 선의 두께를 지정하십시오.
\item[멀리건] 페널티 없이 한 번 더 하는 샷을 멀리건이라고 합니다. 한 홀에서 멀리건을 한 번 사용할 수 있습니다. 한 라운드에서 허용할 멀리건의 수를 지정하십시오.
\item[듀얼 플레이트] 듀얼 플레이트의 기울기 정도를 지정하십시오. \pageref{sec:swing_plate} 쪽 \titleref{sec:swing_plate}\를 참고하십시오.
\item[바람 세기] 바람의 세기를 선택하십시오.
\item[플레이 타입] 득점 방식을 선택하십시오.
    \begin{UI}
    \item[스트로크] 한 라운드에서 가장 적은 누적 타수를 기록한 사람이 우승자가 됩니다.
    \item[스테이블포드] 더블 보기 이하의 성적에 0점이 주어지고, 성적이 그보다 높아질수록 1점씩 가산됩니다. 한 라운드에서 가장 높은 누적 점수를 기록한 사람이 우승자가 됩니다.
    \item[신페리오] 임의의 여섯 홀을 제외하고, 나머지 열두 홀의 성적으로부터 이 계산 방식에 따라 각 참가자의 핸디캡이 산출됩니다. 실제 성적에 핸디캡이 반영되어 우승자가 결정됩니다.
    \item[스킨스] 매 홀마다 승부가 가려집니다.
    \end{UI}
\item[전용 앱 사용] 투비전 앱을 사용할지 선택하십시오. 투비전 앱으로 게임 진행을 제어할 수 있습니다.
\item[퍼팅 이어하기] 게임을 빠르게 진행하려면 이 옵션을 선택하십시오. 모든 플레이어들의 볼들이 그린에 올라와 있을 때, 플레이어들은 자기 차례에서 홀 종료가 이루어질 때까지 (볼이 홀에 들어갈 때까지) 계속 퍼팅할 수 있습니다.
\item[익스트림 골프] 게임을 빠르게 진행하려면 이 옵션을 선택하십시오. 플레이어들은 자기 차례에서 티샷부터 홀 종료가 이루어질 때까지 연속하여 볼을 칠 수 있습니다. 플레이어들은 매 샷마다 주어진 시간 안에 볼을 쳐야 합니다. 제한 시간을 넘기면 1 타의 페널티가 더해집니다. 누적 타수와 사용 시간을 합하여 우승자가 가려집니다.
\item[연습장] 게임을 시작하기 전에 연습 시간을 얼마나 오래 가질지 지정하십시오. 게임을 곧바로 시작하려면 이 옵션을 \ui{OFF}로 설정하십시오. 연습장 기능에 대해 \pageref{chp:practice_range} 쪽 \titleref{chp:practice_range}\를 참고하십시오.
\item[거리 표시] 거리를 어떤 단위로 표시할지 선택하십시오.
\item[그린 거리 표시] 그린에서 거리를 어떤 단위로 표시할지 선택하십시오.
\item[속도 표시] 속도를 어떤 단위로 표시할지 선택하십시오.
\item[코스 및 시작 홀 선택] 어느 홀에서부터 게임을 시작할지 선택하십시오. 코스에서 제외할 홀들을 선택할 수도 있습니다.
\end{UI}

\chapter{게임 진행}

게임이 시작되면 가상 캐디의 음성 안내에 따라 게임을 진행하십시오.
플레이 순서에 관한 일반적인 규칙들은 다음과 같습니다.

\begin{itemize}
\item 첫 홀에서는 명단에 오른 순서대로 티 샷을 합니다.
\item 둘째 샷부터 볼에서 홀까지 거리가 가장 먼 순서대로 샷을 합니다.
\item 그린에서는 볼이 홀에 들어갈 때까지 한 플레이어가 연속하여 볼을 칠 수 있습니다. 이 규칙은 선택 사항입니다.
\item 둘째 홀부터 이전 홀의 성적이 높은 순서대로 티샷을 합니다.
\end{itemize}

\section{준비 자세}

볼을 치기 전에 다음과 같은 방법으로 준비 자세를 취하십시오.

\begin{enumerate}
\item 클럽 헤드를 볼 가까이에 대십시오. 
\item 키오스크의 투비전 센서에 불이 들어오면 볼을 치십시오. 
\end{enumerate}

\image{diagram_shot_ready}

\begin{Warning}
스윙하기 전에 다른 사람들이 멀리 떨어져 있는지 확인하십시오.
듀얼 플레이트가 기울 때 넘어지지 않도록 주의하십시오.
듀얼 플레이트에 두 사람 이상 올라서는 것은 바람직하지 않습니다.
\end{Warning}

\newpage
\section{스크린 영상}

스크린 영상은 현재 위치의 지형과 함께 플레이어를 위한 다양한 정보를 보여줍니다.

\image{screen_view}

화면의 각 구석에 다음과 같은 정보가 표시됩니다.

\begin{callout}*
\item 화면의 왼쪽 위 구석에 홀의 길이와 기본 타수, 그리고 플레이어들의 현재 성적과 홀까지 남은 거리가 표시됩니다.
\item 화면의 오른쪽 위 구석에 홀의 지도가 표시됩니다. 바람의 세기와 방향, 볼에서 목표 지점까지의 거리, 목표 지점에서 홀까지의 거리가 함께 표시됩니다.
\item 화면의 왼쪽 아래 구석에 가상 캐디가 나타나서 샷 준비를 안내합니다.
\item 화면의 오른쪽 아래 구석에 표시되는 매트 이미지가 볼이 어느 매트에 놓여있는지 나타냅니다.
\end{callout}

볼을 치면, 천장과 키오스크에 설치된 투비전 센서가 포착한, 볼이 플레이어의 클럽에 맞는 순간이 재생됩니다.
플레이어는 클럽이 어떤 궤적을 그리며 지나갔는지 그리고 볼이 어느 방향으로 날아갔는지를 재생된 영상에서 확인할 수 있습니다.

\image{screen_impact}

\section{플레이어 기능}

게임 중에 플레이어들이 사용할 수 있는 여러 기능들이 있습니다.
게임 모드, 게임 조건, 볼의 위치에 따라 사용할 수 있는 기능들이 달라집니다.
\lineimg{button_left} 버튼과 \lineimg{button_right} 버튼을 사용하여 플레이어 기능들을 탐색하십시오.

\image{screen_play_functions}

\action{볼을 다시 공급받기}
키패드의 \lineimg{control_button_teeup} 버튼을 누르십시오.

\action{티샷에서 티를 높이기}
키패드의 \lineimg{control_button_up} 버튼을 누르십시오.

\action{티샷에서 티를 낮추기}
키패드의 \lineimg{control_button_down} 버튼을 누르십시오.

\action{볼을 왼쪽으로 겨누기}
키패드의 \lineimg{control_button_left} 버튼을 누르십시오.

\action{볼을 오른쪽으로 겨누기}
키패드의 \lineimg{control_button_right} 버튼을 누르십시오.

\action{더 긴 클럽으로 바꾸기}
키패드의 \lineimg{control_button_up} 버튼을 누르십시오.

\action{더 짧은 클럽으로 바꾸기}
키패드의 \lineimg{control_button_down} 버튼을 누르십시오.

\action{그린에서 퍼팅할 때 화면에 격자 표시하기}
키패드의 \lineimg{control_button_grid} 버튼을 누르십시오. 한 번 더 누르면 격자가 사라집니다.

\image{screen_grid}

\action{다시 샷 하기} 
\ui{멀리건 사용}을 터치하십시오.

\action{플레이어들의 샷 순서를 바꾸기} 
\ui{순서 넘기기}를 터치하십시오.

\action{가상 캐디의 퍼팅 조언 듣기} 
\ui{코스 매니저 사용}을 터치하십시오. 한 라운드에서 이 기능을 세 번 이용할 수 있습니다.

\action{가상 캐디의 조언 다시 듣기} 
\ui{음성 다시 듣기}를 터치하십시오. \ui{코스 매니저 사용} 기능을 쓴 다음에 이 기능을 쓸 수 있습니다.

\action{홀 종료가 이루어질 때까지 연속하여 퍼팅하기} 
\ui{퍼팅 이어하기}를 터치하십시오. 
\pageref{sec:game_options} 쪽 \titleref{sec:game_options}\를 참고하십시오.

\action{현재 위치의 지형 보기} 
\ui{지형 형태 보기/감추기}를 터치하십시오. 현재 위치의 전경을 보여주는 정면 스크린의 화면에 격자가 표시됩니다.

\action{현재 위치에서 홀을 향하여 전경 보기} 
\ui{목표 보기}를 터치하십시오.

\action{현재 위치로부터 특정 지점까지의 거리 보기}
\ui{위치 정보 보기}를 터치하십시오. 정면 스크린의 화면이 터치 모니터에 나타납니다. 한 지점을 터치하십시오. 현재 위치로부터 그 지점까지의 거리가 터치된 지점에 표시됩니다.

\action{그린의 굴곡 보기}
\ui{그린 굴곡 보기}를 터치하십시오. 그린에 등고선이 표시됩니다.

\action{무릎 높이에서 지면 보기}
\ui{시선 낮춰 살펴보기}를 터치하십시오.

\action{홀의 전체 코스 보기}
\ui{둘러보기}를 터치하십시오. 티에서부터 그린까지 비행 영상이 상영됩니다.

\action{1 타 페널티 주기}
 \ui{벌타 드롭}를 터치하십시오.

\action{홀 종료로 인정하기}
\ui{OK 주기}를 터치하십시오. 볼이 그린 위에 올라와 있을 때 홀까지의 거리에 관계없이 이 기능을 사용할 수 있습니다.

\action{현재까지 플레이어들의 성적 보기}
\ui{스코어 카드}를 터치하십시오. \pageref{sec:score_card} 쪽 \titleref{sec:score_card}\를 보십시오.

\action{스윙 동작 보기}
\ui{나스모 보기}를 터치하십시오. 마지막 스윙 동작이 재생됩니다.

\action{마지막 스윙과 이전 스윙을 비교하여 보기}
 \ui{나스모 비교 보기}를 터치하십시오. 마지막 이전 5개까지 스윙 동작이 저장됩니다. 
 비교할 스윙 동작들을 선택하십시오.

\action{대회 참가자들의 순위 보기}
 \ui{실시간 순위 보기}를 터치하십시오.

\action{플레이어 교체하기}
\ui{플레이어 추가/삭제}를 터치하십시오.

\action{현재 홀 경기를 종료하고 다음 홀로 넘어가기}
\ui{홀 넘기기}를 터치하십시오.

\action{가상 캐디를 사용하거나 사용하지 않기}
 \ui{코스 매니저 설정}을 터치하십시오.

\action{왼손잡이 자리 또는 오른손잡이 자리로 바꾸기}
\ui{오른손/왼손 타석 선택}을 터치하십시오.

\action{스윙 동작 영상의 시야각 조정하기}
\ui{시선 조정}을 터치하십시오. 
\pageref{sec:troubleshooting} 쪽 \titleref{sec:troubleshooting}\을 보십시오.

\subsection{스윙 동작 보기}

플레이어의 스윙 동작을 다시 보려면 \ui{나스모 보기}를 터치하십시오.

\image{screen_swing_motion}

최근 다섯 번의 스윙 동작을 볼 수 있습니다.
\ui{Video 1}과 \ui{Video 5} 사이의 버튼을 터치하여 보고 싶은 스윙 동작을 선택하십시오.
천천히 재생하려면 \ui{느리게} 또는 \ui{더 느리게}를 터치하십시오.

\subsection{성적 보기}
\label{sec:score_card}

게임 중에 플레이어들의 성적을 보려면 \ui{스코어 카드}를 터치하십시오. 

\image{screen_score_card}

진행 중인 게임에서 각 플레이어가 기록한 성적을 다양한 통계와 함께 분석할 수 있습니다.
또한 미션 달성률과 배틀존 순위를 확인할 수 있습니다.

\chapter{연습장}
\label{chp:practice_range}

게임을 시작하기 전에 스윙 연습을 할 수 있습니다.
다음 절차를 따라 연습장 기능을 이용하십시오.

\begin{enumerate}
\item 게임 조건을 설정하기 이전 어느 단계의 화면에서 \ui{연습장}을 터치하십시오.
\item 스윙 연습을 하면서 필요에 따라 연습장 기능들을 사용하십시오.
샷을 할 때마다 비거리, 발사각, 볼 속도 등이 화면에 표시됩니다.
\item 마치려면 화면의 오른쪽 상단에 \lineimg{button_exit} 버튼을 터치하십시오.
\end{enumerate}

\image{screen_exercise_menu}

\section{연습장 기능}

정식 게임에서와 동일한 방법으로 키패드를 사용하십시오.

연습장에서 사용할 수 있는 여러 기능들이 있습니다. 연습장 기능들을 사용하려면 \ui{메뉴 보기}를 터치하십시오.

\action{티샷 연습하기} 
\ui{드라이빙 레인지}를 터치하십시오.

\action{어프로치 샷 연습하기} 
\ui{그린 공략 연습장}을 터치하십시오.

\action{퍼팅 연습하기}
\ui{퍼팅 연습장}을 터치하십시오.

\action{클럽 궤적 보기}
\ui{센서 영상 보기}를 터치하십시오. 
볼이 플레이어의 클럽에 맞는 순간이 재생됩니다.
재생된 클럽과 볼의 궤적을 보고 그립 자세와 스윙 자세를 교정하십시오.

\action{나스모 보기}
\ui{나스모 보기}를 터치하십시오. 
키오스크에 설치된 카메라가 촬영한, 플레이어가 스윙하는 순간이 재생됩니다.
재생된 스윙 동작을 보고 스윙 자세를 교정하십시오.

\action{마지막 스윙과 이전 스윙을 비교하여 보기}
\ui{나스모 비교 보기}를 터치하십시오. 최근 다섯 번의 스윙 동작이 저장됩니다. 
비교할 스윙 동작들을 선택하십시오.

\action{연습 기록 보기}
\ui{통계 기록 보기}를 터치하십시오. 

\action{연습 기록 지우기}
\ui{타구 분석 초기화}를 터치하십시오.

\action{클럽별 평균 비거리 구하기}
\ui{클럽별 비거리 측정}을 터치하십시오. 클럽별로 스윙을 10회 반복하십시오. 
이를 통해 산출된 데이터가 정식 게임에서 사용됩니다.
목표 지점까지의 거리에 가장 적합한 클럽이 플레이어에게 추천됩니다.

\begin{Note}
게스트 사용자는 클럽 추천 기능을 사용할 수 없습니다.
\end{Note}

\action{마지막 샷을 연습 기록에서 제외하기}
\ui{샷 취소}를 터치하십시오.

\action{다른 사람들의 비거리 기록 보기}
\ui{장타 대회}를 터치하십시오.

\action{다른 사람들의 홀 근접 기록 보기}
\ui{니어핀 대회}를 터치하십시오.

\action{스윙 동작 영상의 시야각 조정하기}
\ui{시선 조정}을 터치하십시오. 
\pageref{sec:troubleshooting} 쪽 \titleref{sec:troubleshooting}\을 보십시오.

\action{왼손잡이 자리 또는 오른손잡이 자리로 바꾸기}
\ui{오른손/왼손 타석 선택}을 터치하십시오.

\action{몸푸는 방법 보기}
\ui{골프레칭}을 터치하십시오.

\action{볼 아이템 미리보기}
화면의 오른쪽 상단에 \ui{볼 선택}을 터치하십시오.
구매할 수 있는 볼 아이템들의 효과를 경험할 수 있습니다.

\section{연습장 옵션 설정하기}

연습장 옵션을 변경하려면 다음과 같이 하십시오.

\begin{enumerate}
\item 화면의 오른쪽 상단에 \lineimg{button_setting} 버튼을 터치하십시오.
\item 아래에 제시된 설명을 참고하여 옵션들을 적절하게 설정하십시오.
\item 마치려면 \ui{확인}을 터치하십시오.
\end{enumerate}

\image{screen_exercise_settings}

\begin{UI}
\item[카메라 모드] \ui{이동}이 선택되면, 공을 쳤을 때 날아가는 볼을 따라 경치를 감상할 수 있습니다.
\item[볼 이동 경로 표시] \ui{켜기}가 선택되면, 앞서 친 볼들의 궤적이 스크린에 표시됩니다.
\item[그린 상태] 그린의 빠르기를 선택하십시오.
\item[표시 여부] \ui{켜기}가 선택되면, 볼을 칠 때마다 스윙 동작이 재생됩니다.
\item[클럽별 표시] 선택된 클럽을 사용한 경우에만 스윙 동작이 재생됩니다.
\item[듀얼 플레이트 경사 조절] 화살표 버튼들을 사용하여 듀얼 플레이트의 기울기를 조절하십시오.
\end{UI}


\chapter{관리자 설정}

관리자 메뉴를 이용하여 다음과 같은 것들을 할 수 있습니다.

\begin{itemize}
\item 각 게임 옵션의 디폴트 설정
\item 각 장치 옵션의 디폴트 설정
\item 화면 조정
\item 관리자 비밀번호의 변경
\item 각 장치의 하드웨어 설정과 테스트
\end{itemize}

\begin{Caution}
관리자 메뉴 중 \ui{하드웨어 설정}은 서비스 기술자를 위한 것입니다.
하드웨어 설정을 임의로 변경하지 마십시오. 투비전 플러스가 제대로 작동하지 않을 수 있습니다.
\end{Caution}

관리자 설정을 변경하려면 다음 절차를 따르십시오.

\begin{enumerate}
\item 관리자 화면으로 들어가기 위해 투비전 플러스의 시작 화면에서 \lineimg{button_down} 버튼을 터치한 다음에 \ui{관리자}를 터치하십시오. 초기 비밀번호는 1234입니다.
\item 변경할 설정 그룹을 선택하십시오.
\item 옵션들을 적절하게 설정하십시오.
\item 마치려면 \ui{나가기}를 터치하십시오.
\end{enumerate}

\section{스크린 영상 정렬하기}

정면 스크린의 영상과 바닥 스크린의 영상이 세로로 정렬되도록 스크린 영상을 왼쪽 또는 오른쪽으로 옮길 수 있습니다. 다음 절차를 따르십시오.

\begin{IllustEnum}*{screen_admin_screen}
\item 관리자 화면의 오른쪽 하단에 \ui{스크린 정렬}을 터치하십시오.
\item 화살표 버튼들을 사용하여 영상의 위치를 조정하십시오. 버튼을 한 번 누를 때마다 이동 거리를 늘리려면 \ui{+1} 또는 \ui{+10}을 터치하십시오.
\item 마치려면 \ui{확인}을 터치하십시오.
\end{IllustEnum}

\section{사용자 인터페이스 조정하기}

두 층의 영상이 겹쳐 스크린에 영사됩니다.
한 층은 골프 코스의 전경을 보여주고, 다른 층은 사용자 인터페이스로서 미니맵을 포함하여 플레이어를 위한 정보를 보여줍니다.
사용자 인터페이스의 크기와 위치를 조정할 수 있습니다. 다음 절차를 따르십시오.

\begin{IllustEnum}*{screen_admin_gui}
\item 관리자 화면의 오른쪽 하단에 \ui{GUI 화면 설정 }을 터치하십시오.
\item \ui{표시 영역}의 화살표 버튼들을 사용하여 사용자 화면의 크기를, \ui{표시 위치 이동}의 화살표 버튼들을 사용하여 사용자 화면의 위치를 조정하십시오.
\item 마치려면 \ui{확인}을 터치하십시오.
\end{IllustEnum}
    
\section{디폴트 게임 옵션}
\label{sec:admin_setting_default}

각 게임 옵션의 디폴트를 변경할 수 있습니다. 다음 절차를 따르십시오.

\begin{enumerate}
\item \ui{게임 설정}을 터치하십시오.
\item 아래에 제시된 설명을 참조하여 옵션들을 적절하게 설정하십시오. 이전 설정으로 되돌리려면 \ui{되돌리기}를 터치하십시오.
\item 마치려면 \ui{저장}을 터치하십시오. 
\end{enumerate}

\image{screen_admin_game}

\begin{UI}
\item[난이도] 플레이어 수준을 설정하십시오. 선택된 수준에 따라 비거리를 비롯한 여러 요소들의 난이도가 달라집니다.
\item[핀 위치] 그린에서 홀이 위치할 자리를 지정하십시오.
\item[그린 위치] 두 곳의 그린 중 어느 쪽을 사용할지 선택하십시오.
\item[그린 상태] 그린의 빠르기를 선택하십시오.
\item[컨시드] 그린에서 홀로부터 특정한 거리 안에 볼이 이르렀을 때 볼이 홀에 들어간 것으로 인정하는 것을 컨시드라고 합니다. 컨시드 거리를 지정하십시오.
\item[멀리건] 페널티 없이 한 번 더 하는 샷을 멀리건이라고 합니다. 한 홀에서 멀리건을 한 번 사용할 수 있습니다. 한 라운드에서 허용할 멀리건의 수를 지정하십시오.
\item[오비티 사용 설정] \ui{없음}이 선택되면, 오비가 발생했을 때 원래 지점에서 다시 쳐야 합니다. \ui{있음}이 선택되면 오비가 발생한 지점에서 볼을 칠 수 있습니다.
\item[익스트림 골프] 한 샷의 제한 시간을 설정하십시오.
\item[퍼팅 이어하기] 게임을 빠르게 진행하려면 이 옵션을 선택하십시오. 모든 플레이어들의 볼들이 그린에 올라와 있을 때, 플레이어들은 자기 차례에서 홀 종료가 이루어질 때까지 계속 퍼팅할 수 있습니다.
\item[바람 세기] 바람의 세기를 선택하십시오.
\item[거리 표시] 거리를 어떤 단위로 표시할지 선택하십시오.
\item[그린 거리 표시] 그린에서 거리를 어떤 단위로 표시할지 선택하십시오.
\item[속도 표시] 속도를 어떤 단위로 표시할지 선택하십시오.
\item[퍼팅 격자] 퍼팅이 가능할 때 화면에 격자가 표시됩니다. 격자 선의 두께를 지정하십시오.
\item[더블파 모드] \ui{켜기}가 선택되면, 한 홀에서 어떤 플레이어가 기준 타수의 두 배를 기록했다면 그것이  그 사람의 최종 홀 성적이 됩니다. 빠른 진행을 위해 \ui{켜기}로 설정하는 것이 좋습니다.
\item[퍼팅 가이드 LED 설정] 퍼팅 가이드를 사용할지 선택하십시오. 퍼팅 가이드에 대해 \pageref{sec:putting_guide} 쪽 \titleref{sec:putting_guide}\를 참고하십시오.
\end{UI}

\section{디폴트 장치 옵션}

오토티업기를 비롯하여 장치들과 연관된 옵션들의 디폴트를 변경할 수 있습니다. 다음 절차를 따르십시오.

\begin{enumerate}
\item \ui{시스템 설정}을 터치하십시오.
\item 아래에 제시된 설명을 참조하여 옵션들을 적절하게 설정하십시오. 이전 설정으로 되돌리려면 \ui{되돌리기}를 터치하십시오.
\item 마치려면 \ui{저장}을 터치하십시오. 
\end{enumerate}
    
\image{screen_admin_system}

\begin{UI}
\item[티 높이] 티의 높이를 지정하십시오.
\item[듀얼 플레이트] 듀얼 플레이트의 기울기 정도를 지정하십시오.
\item[타석 위치] 오른손잡이용 듀얼 플레이트가 설치되어 있다면 \ui{오른손}을 선택하십시오. 
\item[카메라 모드] \ui{중계}가 선택되면, 공을 쳤을 때 날아가는 볼을 따라 경치를 감상할 수 있습니다.
\item[화면 해상도] 스크린의 해상도를 지정하십시오.
\item[바닥 스크린] 바닥 스크린을 사용할지 선택하십시오.
\item[매트 타입] 투어 러프 매트와 투어 벙커 매트가 설치되어 있으면 \ui{5종 매트}를 선택하십시오.
\item[일반 라운딩] \ui{켜기}가 선택되면, 게임 중에 볼을 칠 때마다 스윙 동작이 재생됩니다.
\item[연습장 모드] \ui{켜기}가 선택되면, 연습 중에 볼을 칠 때마다 스윙 동작이 재생됩니다.
\item[클럽별 표시] 선택된 클럽을 사용한 경우에만 스윙 동작이 재생됩니다.
\item[사운드 모드] 5.1 채널 또는 스테레오를 선택하십시오.
\end{UI}

\section{관리자 비밀번호 바꾸기}

관리자 비밀번호와 사업 운영에 관계된 설정을 변경할 수 있습니다. 다음 절차를 따르십시오.

\begin{enumerate}
\item \ui{GS 관리 설정}을 터치하십시오.
\item 아래에 제시된 설명을 참조하여 옵션들을 적절하게 설정하십시오. 이전 설정으로 되돌리려면 \ui{되돌리기}를 터치하십시오.
\item 마치려면 \ui{저장}을 터치하십시오. 
\end{enumerate}

\image{screen_admin_gs}

\begin{UI}
\item[관리자 비밀번호] 새 비밀번호를 입력하십시오.
\item[비밀번호 확인] 비밀번호 입력을 요구하는 상황들을 선택하십시오.
\item[CC 랭킹 초기화] 각 골프 코스의 순위 기록을 삭제하십시오.
\item[라운드 독려 메시지] \ui{사용}이 선택되면, 게임이 지연되어 일정 시간이 경과했을 때 경기를 독려하는 메시지가 나옵니다.
\end{UI}

\chapter{문제 해결과 유지 관리}

\section{소모품}

다음 품목들은 소모품입니다. 

\begin{itemize}
\item 오토티업기의 고무 티
\item 페어웨이 매트를 비롯한 모든 매트
\item 정면 스크린과 바닥 스크린
\item 프로젝터 램프와 필터
\end{itemize}

소모품이 심하게 손상되거나 마모되었으면 골프존에 소모품 교체를 요청하십시오.

\section{문제 해결}
\label{sec:troubleshooting}

각 문제에 대해 주어진 절차에 따라 해결을 시도하십시오.
시도한 방법으로 증상이 사라지지 않고 문제가 지속되면 골프존에 서비스를 요청하십시오.

\problem{프로젝터의 LAMP 지시등은 켜져있지만 프로젝터가 작동하지 않는다}
램프가 제대로 장착되어 있는지 확인하십시오.

\problem{프로젝터의 TEMP 지시등은 켜져있지만 프로젝터가 작동하지 않는다}
프로젝터 내부 온도가 갑자기 오르면 이와 같은 증상이 발생합니다.
프로젝터를 끄고, 프로젝터가 충분히 식으면 프로젝터를 다시 켜십시오.

\problem{프로젝터들이 켜져있지만 스크린에 아무 것도 영사되지 않는다}

\begin{enumerate}
\item 투비전 플러스 프로그램을 종료하십시오.
\item 프로젝터들이 켜져있는지 확인하십시오.
\item 투비전 플러스 프로그램을 실행하십시오.
\end{enumerate}

\problem{스윙 동작이 재생될 때 영상이 선명하지 않다}
키오스크의 전면에 장착된 카메라의 포커스 링을 돌려서 초점을 적절하게 맞추십시오.

\problem{스윙 동작의 일부만 재생된다}
카메라 타이밍이 제대로 설정되어 있지 않습니다. 골프존에 서비스를 요청하십시오.

\problem{스윙 동작이 재생될 때 시야각이 실제와 다르다}

\begin{enumerate}
\item 게임 메뉴 또는 연습장 메뉴에서 \ui{시선 조정}을 터치하십시오.
\item 볼을 페어웨이 매트에 올려놓고 \ui{페어웨이 매트 시선 조정}의 화살표 버튼들을 사용하여 카메라의 시야각을 조정하십시오.
\item 볼을 러프 매트와 벙커 매트에 올려놓고 \ui{러프/벙커 매트 시선 조정}의 화살표 버튼들을 사용하여 카메라의 시야각을 조정하십시오.
\end{enumerate}

\problem{준비 자세를 취했으나 키오스크의 투비전 센서에 불이 들어오지 않는다}
키오스크를 끄고 다시 켜십시오.

\problem{볼을 쳤을 때 볼이 지나치게 느리게 날아간다}
키오스크를 끄고 다시 켜십시오.

\problem{회원 카드로 로그인을 할 수 없다}
카드 리더기에 주황색 지시등이 켜져있는지 확인하고 다시 시도하십시오. 
지시등이 꺼져있다면 골프존에 서비스를 요청하십시오.

\newpage
\section{진단}

투비전 센서가 정상적으로 작동하는지 점검하기 위한 진단 프로그램이 제공됩니다. 

\begin{flushleft}
\path{C:\Program Files\Golfzon\GolfzonVision2\Bin64} 폴더에서 \path{NGSSensorDiag.exe} 파일을 실행하십시오.
\end{flushleft}

\image{screen_ngssensordiag}

다음 절차를 따라 투비전 센서를 진단하십시오.

\begin{enumerate}
\item 진단 옵션들을 적절하게 설정하십시오.
    \begin{itemize}
    \item \ui{시스템 설정}을 실제 설치에 맞게 설정하십시오.
    \item \ui{시스템 설정}에 \ui{양손}을 설정했다면, \ui{타석 설정}을 진단할 자리로 설정하십시오.
    \item \ui{동작모드 설정}을 \ui{진단도구}로 설정하십시오.
    \end{itemize}
\item \ui{Start} 버튼을 클릭하십시오.
\item 반복하여 볼을 치십시오. 볼 속도를 비롯한 인식 결과가 창의 오른쪽에 표시됩니다. 오른쪽 위 구석에 표시된 숫자는 샷 횟수를 나타냅니다.
    \begin{itemize}
    \item \ui{매트설정}을 다르게 설정하고, 선택한 매트에 볼을 놓고 치십시오.
    \item \ui{클럽설정}을 다르게 설정하고, 선택한 클럽으로 볼을 치십시오.
    \end{itemize}
\item 마치려면 \ui{Stop} 버튼을 클릭하십시오.
\end{enumerate}

\subsection{듀얼 플레이트 조작}

진단 중에 듀얼 플레이트를 조정해야 한다면 \ui{듀얼 플레이트 설정} 버튼을 클릭하십시오.

\image{screen_ngssensordiag_swing_plate.png}

\begin{enumerate}
\item \ui{포트 검색} 버튼을 클릭하십시오. 진단 프로그램이 듀얼 플레이트를 찾아 접속합니다.
\item \ui{경사도 1}과 \ui{경사도 2}의 슬라이드바를 사용하여 경사도를 지정하고 \ui{경사도 설정} 버튼을 클릭하십시오. 듀얼 플레이트가 지정된 각도로 기울어집니다.
듀얼 플레이트를 원래의 평평한 모양으로 되돌리려면 \ui{1원점 이동} 버튼을 클릭하십시오.
\item 듀얼 플레이트 조작을 마치려면 \ui{포트 닫기} 버튼을 클릭하십시오. 진단 프로그램과 듀얼 플레이트 사이의 접속이 해제됩니다.
\end{enumerate}

\subsection{퍼팅 조정}

퍼팅했을 때 볼이 한 쪽으로 치우치는 경향을 바로잡을 수 있습니다.
\menu{설정}{퍼팅 방향 조정} 메뉴를 선택하십시오.

\image{screen_ngssensordiag_putting.png}

\begin{enumerate}
\item \ui{센서 위치} 옵션을 실제 설치에 맞게 설정하십시오.
\item 볼이 오른쪽으로 치우치면 \ui{좌(-)} 버튼을, 왼쪽으로 치면 \ui{우(+)} 버튼을 클릭하여 보정값을 조정하십시오.
\item 마치려면 \ui{닫기} 버튼을 클릭하십시오.
\end{enumerate}

\chapter{보증 및 서비스}

\section{고객 지원 센터}

아래 연락처를 이용하여 ㈜골프존에 수리 서비스를 요청하십시오.

\begin{terms}|
\item[전화] 1577-4333
    \begin{terms}|
    \item[평일] 09:00\,--\,23:00
    \item[주말 및 공휴일] 09:00\,--\,19:00
    \end{terms}
%\item[인터넷] \url{http://glm.golfzon.com}
\item[이메일] \email{webmaster@golfzon.com}
\end{terms}

\section{보증}

보증 범위와 기간은 ㈜골프존과 가맹주 사이에 체결된 계약에 따릅니다.
다음과 같은 경우에 수리 서비스가 무료로 제공됩니다.

\begin{itemize}
\item 정상적인 사용 상태에서 보증 기간 내에 발생한 고장
\item 수리한 날로부터 2개월 안에 발생한 동일한 고장
\end{itemize}

\section{책임 한도}

다음의 경우에는 보증 수리가 적용되지 않습니다.

\begin{itemize}
\item 가맹 계약이 종료되었거나 해지되었을 때
\item 허용되지 않는 방법으로 사용하여 발생한 고장
\item 부식, 낙하, 침수를 포함하여 잘못된 보관으로 인해 발생한 고장
\item 가맹 사업자를 포함하여, 골프존으로부터 인가받지 않은 자가 제품을 변경하거나 분해하거나 수리한 뒤에 발생한 고장 (변조 방지를 위한 봉인 라벨이 훼손되었을 때)
\item 태풍, 홍수, 지진, 낙뢰, 이상 전압을 포함하여 천재지변이나 불가항력에 의해 발생한 고장
\item 매트, 스크린, 고무 티를 포함하는 소모성 부품
\item 소모품 교체를 포함하여 사업자가 할 수 있는 조치에 대해 서비스를 요청했을 때
\item 소비자의 부주의로 인해 발생한 고장
\item 제공된 것 이외에 다른 소프트웨어가 사용되었을 때
\item 디컴파일이나 그와 유사한 방법을 이용하여 소프트웨어를 변조했을 때
\end{itemize}

\chapter{안전 지침}

\section{접지}

\begin{Warning}
접지를 절대로 제거하거나 훼손하지 마십시오. 접지가 제거되거나 훼손되어 발생한 고장에 대해서는 보증 수리가 적용되지 않습니다. 또한 그로 인해 발생한 어떠한 사고에 대해서도 골프존은 책임을 지지 않습니다.
\end{Warning}

접지는 가장 기본적인 안전 장치입니다. 접지가 제대로 되어 있지 않으면 감전 사고나 제품의 오작동이 발생할 수 있습니다. 
심지어 과부하나 다른 심각한 문제가 발생했을 때 전기 차단기가 정상적으로 작동하지 못하여 화재가 발생할 수 있습니다.
인명 사고와 제품 손상을 예방하기 위하여 투비젼 플러스 설비에는 다음과 같이 접지가 설치되어 있습니다.

\begin{SpecTable}
종류 & 제3종  \\
접지 저항 & 100 Ω 이하 \\
접지선 굵기 & 1.6 mm 이상 \\
용도 & 400 V 미만 저압용 설비 \\
\end{SpecTable}

\section{전원}

\begin{Warning}
전기 설비를 임의로 개조하지 마십시오. 
부적절한 전기 설비로 인해 발생한 고장에 대해서는 보증 수리가 적용되지 않습니다. 또한 그로 인해 발생한 어떠한 사고에 대해서도 골프존은 책임을 지지 않습니다.
\end{Warning}

화재 또는 감전 사고를 예방하기 위하여 다음 사항을 주의하십시오.

\begin{itemize}
\item AC 200--240 V 정격 전원을 사용하십시오.
\item 각 장비마다 개별 차단기가 설치되어 있어야 합니다. 장비당 20 A 이상의 차단기를 사용해야 합니다.
\item 투비전 플러스 설비당 최대 사용 전력은 약 2000 W입니다. 에어컨을 비롯한 다른 전기 기구의 소비 전력을 감안하여 전기 설비를 유지하고 관리하십시오.
\item 손상된 플러그와 헐거운 콘센트를 사용하지 마십시오. 
\item 변형되거나 손상된 전선을 사용하지 마십시오.
\item 물이나 다른 액체가 설비에 유입되었으면 전원을 차단하고 골프존에 서비스를 요청하십시오.
\end{itemize}

\section{관리}

\begin{Caution}
투비전 플러스의 인테리어를 임의로 변경하지 마십시오.
인테리어 변경으로 인해 발생한 성능 저하나 고장에 대해서는 보증 수리가 적용되지 않습니다.
\end{Caution}

제품의 오작동과 고장을 예방하기 위하여 다음 사항을 주의하십시오.

\begin{itemize}
\item 듀얼 플레이트로부터 천장까지 높이를 2800\,--\,3200 mm 이내로 유지해야 합니다.
\item 습도를 낮게 유지하십시오.
\item 먼지 유입을 차단하고 자주 환기하십시오.
\item 실내 온도를  0\,--\,40 °C 이내로 유지하십시오.
\item 100 Mbps를 기준으로 한 회선에 연결된 장비의 수는 5 대 이하이어야 합니다.
\item 직사광선을 차단하십시오.
\item 프로젝터 광선을 똑바로 바라보지 마십시오. 특히 무수정체증, 백색증, 백내장 환자는 프로젝터 광선에 노출되지 않도록 주의하십시오.
\end{itemize}

\section{사용}

투비전 플러스를 사용할 때 다치지 않도록 다음 사항을 주의하십시오.

\begin{itemize}
\item 스윙하기 전에 준비 운동을 충분히 하십시오.
\item 스윙하기 전에 다른 사람이나 물건이 주위에 없는지 확인하십시오.
\item 스윙할 때 듀얼 플레이트의 뒤쪽 가장자리를 치지 않도록 주의하십시오.
\item 듀얼 플레이트가 움직일 때 넘어지지 않도록 주의하십시오.
\end{itemize}

\appendix

\chapter{사양}

\subsection{키오스크}

\begin{SpecTable}
운영 체제 & Windows 7 \\
터치 모니터 해상도 & 1920 × 1080 \\
카메라 프레임 속도 & 60 FPS \\
블루투스 & 4.0 \\
\end{SpecTable}

\subsection{듀얼 플레이트}

\begin{SpecTable}
크기 & 1807 mm × 1307 mm × 190--235 mm \\
타석부 최대 허용 하중 &  150 kg \\
타격부 최대 허용 하중 &  80 kg \\
타석부 최대 기울기 & 4° \\
타격부 최대 기울기 & 7° \\
페어웨이 매트 & 800 mm × 350 mm \\
투어 러프 매트 & 400 mm × 230 × 87/65 mm \\
투어 벙커 매트 & 400 mm × 230 × 82 mm \\
키패드 & 403 mm × 176 mm × 38 mm \\
퍼팅 가이드 & 350 mm × 7 mm × 13 mm \\
사용 온도 & 0--40 °C \\
사용 습도 & 20--80\% \\
\end{SpecTable}

\subsection{정면 프로젝터}

\begin{SpecTable}
모델 & Hitachi CP-F650 \\
밝기 & 6000 lm \\
해상도 & 1920 × 1200 \\
화면 비율 & 1.29:1 \\
크기 & 498 mm × 135 mm × 396 mm \\
무게 & 8.9 kg \\
소비 전력 & 500 W \\
\end{SpecTable}

\subsection{정면 프로젝터 (고사양)}

\begin{SpecTable}
모델 & Sony VPL-PHZ10 \\
밝기 & 5000 lm \\
해상도 & 1920 × 1200 \\
화면 비율 & 1.28:1 \\
크기 & 510 mm × 113 mm × 354 mm \\
무게 & 8.7 kg \\
소비 전력 & 403 W \\
\end{SpecTable}


\subsection{바닥 프로젝터}

\begin{SpecTable}
모델 & Hitachi CP-302WN \\
밝기 & 3200 lm \\
해상도 & 1280 × 800 \\
화면 비율 & 0.59:1 \\
크기 & 350 mm × 134 mm × 305 mm \\
무게 & 3.6 kg \\
소비 전력 & 320 W \\
\end{SpecTable}

\subsection{바닥 프로젝터 (고사양)}

\begin{SpecTable}
모델 & NEC NP-P554U \\
밝기 & 5600 lm \\
해상도 & 1920 × 1200 \\
화면 비율 & 0.65:1 \\
크기 & 420 mm × 133 mm × 322 mm \\
무게 & 6.5 kg \\
소비 전력 & 431 W \\
\end{SpecTable}

\subsection{5.1 채널 스피커}

\begin{SpecTable}
채널 & 5.1 \\
크기 & 92 mm × 97 mm × 153 mm \\
\end{SpecTable}

\chapter{용어}

\begin{terms}*[base={티잉 그라운드}]
\item[그린] green. 골프 코스에서 홀 있는 곳으로, 고운 잔디로 정비되어 있다.
\item[그립] grip. 손으로 잡는 클럽의 한쪽 끝 부분 또는 그 구분을 잡는 것.
\item[더블 보기] double bogey. 파보다 2 타 많은 타수.
\item[디봇] divot. 골프 클럽에 의해 뜯겨진 한 조각의 잔디.
\item[라운드] round. 18 홀로 이루어진 한 회의 경기.
\item[러프] rough. 페어웨이의 가장자리. 다듬어지지 않은 잔디나 잡초로 이루어져 있다.
\item[멀리건] mulligan. 페널티 없이 한 번 더 볼을 치는 것.
\item[버디] birdie. 파보다 1 타 적은 타수.
\item[벙커] bunker. 모래로 채워진 웅덩이.
\item[보기] bogey. 파보다 1 타 많은 타수.
\item[알바트로스] albatross. 파보다 3 타 적은 타수.
\item[어프로치] approach. 티샷 이후 그린
\item[에이프런] apron. 그린의 가장자리. 에이프런의 잔디는 페어웨이보다 짧고 그린보다 길다.
\item[오비] OB. out of bounds의 줄임말로, 볼이 코스에서 완전히 벗어난 것을 의미한다.
\item[이글] eagle. 파보다 2 타 적은 타수.
\item[컨시드] concede. 그린에서 홀로부터 특정한 거리 안에 볼이 이르렀을 때 볼이 홀에 들어간 것으로 인정하는 것
\item[쿼드러플 보기] quadruple bogey. 파보다 4 타 많은 타수.
\item[클럽] club. 골프 볼을 치기 위해 사용하는 봉.
\item[트리플 보기] triple bogey. 파보다 3 타 많은 타수.
\item[티] tee. 볼을 받치는 작은 막대기. 티샷을 할 때만 티를 사용할 수 있다.
\item[티샷] tee shot. 티잉 그라운에서 하는 첫 번째 샷.
\item[티잉 그라운드] teeing ground. 한 홀의 시작 지점. 티잉 그라운드에서 티샷을 함으로써 한 홀의 경기가 시작된다.
\item[파] par. 각 홀의 기준 타수 또는 그와 동일한 성적.
\item[퍼팅] putting. 홀에 넣기 위해 볼을 치는 것.
\item[페널티] penalty. 볼이 코스에서 벗어나거나 물 웅덩이에 빠졌을 때 플레이어의 타수에 하나를 더하는 것. 페널티를 받은 뒤에 원래 위치에서 다시 볼을 치는 것이 원칙이다.
\item[페어웨이] fairway. 티잉 그라운드에서 그린까지 이어지는 잔디밭. 잔디가 짧게 잘 다듬어져 있어 볼을 치기에 좋다.
\item[핀] pin. 홀에 꽂는 푯대
\item[해저드] hazard. 벙커나 물 웅덩이 같은 장애물.
\item[홀] hole. 목표 지점으로서 그린에 위치한 구멍. 골프 코스는 일반적으로 18 홀을 한 단위로 구성된다.
\end{terms}

\PrintIndex

\BackCover

\end{document}
